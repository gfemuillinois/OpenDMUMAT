% Created 2024-09-04 Wed 15:30
% Intended LaTeX compiler: pdflatex
\documentclass[11pt]{article}
\usepackage[utf8]{inputenc}
\usepackage[T1]{fontenc}
\usepackage{graphicx}
\usepackage{longtable}
\usepackage{wrapfig}
\usepackage{rotating}
\usepackage[normalem]{ulem}
\usepackage{amsmath}
\usepackage{amssymb}
\usepackage{capt-of}
\usepackage{hyperref}
\usepackage[margin=0.5in]{geometry}
\usepackage{bpmResearchPaper}
\setlength{\parindent}{0pt}
\author{Bryce Mazurowski}
\date{\today}
\title{OpenDM Two Parameter UMAT: C++ Manual}
\hypersetup{
 pdfauthor={Bryce Mazurowski},
 pdftitle={OpenDM Two Parameter UMAT: C++ Manual},
 pdfkeywords={},
 pdfsubject={},
 pdfcreator={Emacs 29.4 (Org mode 9.6.15)}, 
 pdflang={English}}
\begin{document}

\maketitle

\section{Theory}
\label{sec:orgdf29f96}
\subsection{Problem}
\label{sec:org8530420}
\subsubsection{Solid mechanics: infinitesimal deformation}
\label{sec:org5b776cb}
Strong Form neglecting inertial effects and body forces:
\begin{equation}
  \diverge \stress = 0 \quad \forall \pos \in \Omega
\end{equation}

BCs:
\begin{equation}
  \begin{split}
    \disp &= \bar{\disp} \quad \forall \pos \in \Gamma_d \\
    \stress \dot \vect{n} &= \bar{\traction}
    \quad \forall \pos \in \Gamma_n \\
  \end{split}
\end{equation}

Strain:
\[
\strain = \frac{1}{2}(\grad \disp + \grad^T \disp)
\]

\subsubsection{Weak form}
\label{sec:org3927006}
Weak Form:
\[
\int_{\Omega} \grad \weight : \stress d\Omega =
\int_{\Gamma_n} \weight \dot \bar{\traction} d\Gamma
\]

\subsubsection{Material nonlinearity:}
\label{sec:orgeadafd3}
Material nonlinearity requires a special solver. Newton-Raphson is the
classical choice. However, as discussed in Section\textasciitilde{}\ref{sec:org36dc872}. The displacement solution becomes incremental
\[
\disp_n = \sum_{j=0}^{n} \Delta \disp
\]

With updates at each NR solve at each load step:
\[
\disp_{n+1} = \disp_n + \Delta \disp_{n+1}
\]

Make weak form a residual:
\[
R = 
\int_{\Gamma_n} \weight \dot \bar{\traction} d\Gamma -
\int_{\Omega} \grad \weight : \stress d\Omega
\]

Linearize with Taylor Series:
\[
R (\disp_{n+1}) = R (\disp_{n}) +
\frac{\partial R}{\partial \disp} (\disp_{n})
\delta \disp_{n+1} +
HOT
\]

Progressing from a converged solution, \(R(\disp_n) = 0\), dropping
higher order terms
\[
\frac{\partial R}{\partial \disp}(\disp_n) \delta \disp_{n+1} \approx 
R (\disp_{n+1})
\]

The quantity \(\disp_{n+1}\) is found iteratively in the Newton-Raphson
scheme. So we solve a given load step \(n\) for a number of NR
increments \(i\) as:
\[
\frac{\partial R}{\partial \disp} (\disp_n^i) \delta \disp_{n+1}^{i+1} =
R (\disp_{n}^{i})
\]

With our residual, this becomes:
\[
\int_{\Omega} \grad \weight : \frac{\partial \stress}{\partial \disp}
\delta \disp_{n+t}^{i+1} d\Omega =
\int_{\Gamma_n} \weight \dot \bar{\traction} -
\int_{\Omega} \grad \weight : \stress (\disp) d\Omega 
\]

Where stress depends on \(\disp_{n}^{i}\) in each term. Once the system
of equations is solved for a give \(i\), the displacement is updated. The
residual is then recomputed, and can be used as a convergence check.

The term \(\frac{\partial \stress}{\partial \disp}\) is the material
tangent stiffness matrix.
\subsubsection{{\bfseries\sffamily TODO} Relevant coordinate systems and transformations}
\label{sec:org98c4a7f}
\begin{itemize}
\item Problem coordinates
\item Material coordinates
\item Transforming between the two
\item Notation for each case
\end{itemize}
\subsection{OpenDM Two Parameter Model}
\label{sec:orgc376f26}
\subsubsection{Physical motivation}
\label{sec:orgd3feb51}
\begin{itemize}
\item Damage formulation for woven CMCs (Cite Maire papers)
\item Damage parameters represent cracking
\begin{itemize}
\item d1: Aligned with principal direction of mat coords
\item d2: Perpendicular to principal direction of mat coords
\end{itemize}
\end{itemize}
\begin{enumerate}
\item {\bfseries\sffamily TODO} Add figure
\label{sec:org527de92}
\item {\bfseries\sffamily TODO} Add citation
\label{sec:org10ac20b}
\end{enumerate}
\subsubsection{Constitutive behavior}
\label{sec:org56e2ece}
\begin{enumerate}
\item Helmholtz free energy
\label{sec:org2452ccf}

\[
\psi = \frac{1}{2 \rho} (\strain^* : \tilde{\vect{C}} : \strain^*)
\]

Where \(\strain^*\) is mechanical strain:
\(\strain^* = \strain - \strain^{th}\)
\item Stress
\label{sec:orge0cf328}

Stress is the derivative of the Helmholtz free energy with respect to strain:
\[
\stress = \rho \frac{\partial \psi}{\partial \strain} =
\tilde{\vect{C}} : \strain^*
\]

\item Material stiffness matrix
\label{sec:org5d7b656}

The material stiffness tensor is affected by the two damage parameters
and their progression. This relationship depends on the strain and is
calculated as follows:
\[
\tilde{\vect{C}} = (\tilde{\vect{S}})^{-1}
\]

where \(\tilde{\vect{S}}\) is the material compliance tensor for a given
strain and history.

The material compliance tensor for OpenDM two parameter model is
calculated as follows:
\[
\tilde{\vect{S}} = \vect{S}^0 + d^1 \vect{H}^1 + d^2 \vect{H}^2
\]

\(\vect{S}^0\) is the undamaged material compliance tensor.

\(\vect{H}^i\) tensors are constants that modify the material
compliance for each damage component. They are calculated as:
\[
\vect{H}^1 =
\begin{bmatrix}
  h_1^1 S_{11}^0 & 0 & 0 & 0 & 0 & 0 \\
  0 & 0 & 0 & 0 & 0 & 0 \\
  0 & 0 & 0 & 0 & 0 & 0 \\
  0 & 0 & 0 & 0 & 0 & 0 \\
  0 & 0 & 0 & 0 & h_3^1 S_{55}^0 & 0 \\
  0 & 0 & 0 & 0 & 0 & h_2^1 S_{66}^0 
\end{bmatrix}
\]

and
\[
\vect{H}^2 =
\begin{bmatrix}
  0 & 0 & 0 & 0 & 0 & 0 \\
  0 & h_1^2 S_{22}^0 & 0 & 0 & 0 & 0 \\
  0 & 0 & 0 & 0 & 0 & 0 \\
  0 & 0 & 0 & h_3^2 S_{44}^0 & 0 & 0 \\
  0 & 0 & 0 & 0 & 0 & 0 \\
  0 & 0 & 0 & 0 & 0 & h_2^2 S_{66}^0 
\end{bmatrix}
\]

The quantities \(h_i^m\) are mode specific parameters that increase or
decrease the effect of damage has on each mode.

\(\vect{H}^1\) brings stiffness changes as a result of Mode I, II, and
III microcracking in the CMC matrix as \(d^1\) increases from zero. This
corresponds to microcracks that are growing perpendicular to the
principle material direction.

\(\vect{H}^2\) Brings in the same microcracking effects, but for cracks
that are parallel to the principle material direction.

\item Material tangent stiffness matrix
\label{sec:org8d4e77e}

The material tangent stiffness tensor is the derivative of the stress
wrt strain:
\[
\vect{C}_{tan} = \frac{\partial \stress}{\partial \strain} = 
\frac{\partial \tilde{\vect{C}}}{\partial \strain} +
\tilde{\vect{C}}
\]

Material nonlinearity is reflected by the fact that the partial
derivative of the material stiffness tensor with respect to strain is
not zero. In the OpenDM model, this nonlinearity is caused by the two
damage parameters \(d^1\) and \(d^2\)

\item Damage values and evolution
\label{sec:org79bfa6d}

The damage variables evolve based on individual psuedo potentials:
\[
F^i = f(y^i) - d^i
\]

Driving forces \(y^i\) are calculated for each damage mode:
\begin{equation}
  \begin{split}
    y^1 &= \frac{1}{2} (
    \varepsilon_{11} C^0_{1111} \varepsilon_{11} + 
    b_2 \varepsilon_{13} C^0_{1313} \varepsilon_{13} + 
    b_1 \varepsilon_{12} C^0_{1212} \varepsilon_{12} ) \\ 
    y^2 &= \frac{1}{2} (
    \varepsilon_{22} C^0_{2222} \varepsilon_{22} + 
    b_2 \varepsilon_{23} C^0_{2323} \varepsilon_{23} + 
    b_1 \varepsilon_{12} C^0_{1212} \varepsilon_{12} ) \\ 
  \end{split}
\end{equation}

where \(b_i\) are damage specific shear coupling parameters that
increase the amount shear strains drive damage growth.

For convenience, another step is added here:
\[
g^i = \frac{\langle \sqrt{y_{max}^i} - \sqrt{y_0^i}\rangle_+}{
\sqrt{y_c^i}}
\]

where we have added a history effect in \(y_{max}^i\), which 
takes the largest \(y^i\) value over time for each mode at the time of calculation
for a given material point.
\(\langle \dot \rangle_+\) is the positive
Macaulay bracket.

\(y_0^i\) is a model parameter that sets a threshold value of the
driving force to start damage.

\(y_c^i\) is a model parameter that changes the celerity of the damage
progression.

The final step is
\[
f(y^i) = d_c^i [1 - \exp^{-(g^i)^{(p^i)}}]
\]

\(d_c^i\) sets the maximum value of the damage parameter.

\(p^i\) changes the shape of the curve of damage progression.
\item Analytical tangent stiffness matrix
\label{sec:org1ec9315}

Starting from the equation for stress in Voigt notation,
\[
\sigma_i = \tilde{C}_{ik} \varepsilon_{k},
\]

the analytical tangent stiffness matrix can be calculated.

\[
\frac{\partial \sigma_i}{\partial \varepsilon_j} =
\frac{\partial \tilde{C}_{ik}}{\partial \varepsilon_j} \varepsilon_k +
\tilde{C}_{ij}
\]

The difficult part of the above equation is
\(\frac{\partial \tilde{c}_{ik}}{\partial \varepsilon_j}\), which is
calculated as
\[
\frac{\partial \tilde{C}_{ik}}{\partial \varepsilon_{j}} =
\frac{\partial \tilde{S}_{ik}^{-1}}{\partial \varepsilon_{j}}.
\]

The derivative of the inverse of the effective compliance tensor can
be computed if we start from the indentity for the inverse.
\begin{equation}
  \begin{split}
    \tilde{S}_{ik}^{-1} S_{kl} &= \delta_{il} \\
    \frac{\partial}{\partial \epsilon_{j}} (\tilde{S}_{ik}^{-1} S_{kl}) &= 0_{ilj} \\
    \frac{\partial \tilde{S}_{ik}^{-1}}{\partial \varepsilon_{j}} \tilde{S}_{kl} +
    \tilde{S}_{ik}^{-1} \frac{\partial \tilde{S}_{kl}}{\partial \varepsilon_{j}} &= 0_{ilj} \\
    \frac{\partial \tilde{S}_{ik}^{-1}}{\partial \varepsilon_{j}} \tilde{S}_{kl} &=
    -\tilde{S}_{ik}^{-1} \frac{\partial \tilde{S}_{kl}}{\partial \varepsilon_{j}} \\
    \frac{\partial \tilde{S}_{ik}^{-1}}{\partial \varepsilon_{j}} =
    -\tilde{S}_{ik}^{-1} \frac{\partial \tilde{S}_{km}}{\partial \varepsilon_{j}} \tilde{S}_{ml}^{-1} &=
    -\tilde{C}_{ik} \frac{\partial \tilde{S}_{km}}{\partial \varepsilon_{j}} \tilde{C}_{ml}
  \end{split}
\end{equation}

This includes the assumption that the compliance tensor is invertible,
but that is a must have in mechanics. The next step is the derivative
of the compliance tensor.
\[
\frac{\partial \tilde{S}_{km}}{\partial \varepsilon_j} =
\frac{\partial}{\partial \varepsilon_j} (S_{km}^{0} + \sum_{n=1}^{2} d^n H^n_{km}) = \sum_{n=1}^2 \frac{\partial d^n}{\partial \varepsilon_j} H_{km}^n
\]

The derivative for each damage mode can then be calculated
\[
\frac{\partial d^n}{\partial \varepsilon_j} =
\frac{\partial}{\partial \varepsilon_j}
(d_c^n(1 - \exp^{-(g^n)^{(p^m)}})) =
\frac{\partial}{\partial \varepsilon_j}(-d_c^n \exp^{-(g^n)^{(p^n)}})
\]

The only non-constants in this formula is \(g^n\). We can use the chain
rule to simplify this.
\[
\frac{\partial d^n}{\partial \varepsilon_j} = 
\frac{\partial d^n}{\partial g^n} \frac{\partial g^n}{\partial \varepsilon_j}
\]

The derivative with respect to \(g^n\) is then
\[
\frac{\partial d^n}{\partial g^n} = d_c^n p^n (g^n)^(p^n - 1) \exp^{-(g^n)^{p^n}}
\]

Now, looking at \(g^n\).
\[
\frac{\partial g^n}{\partial \varepsilon_j} =
\frac{\partial}{\partial \varepsilon_j}
(\frac{\langle \sqrt{y_{max}^n} - \sqrt{y_0^n} \rangle_+}{\sqrt{y_c^n}})
\]

In this case, there is again only one term that depends on \(\strain\),
\(y_{max}^n\). Again the chain rule can be exercised.
\[
\frac{\partial g^n}{\partial \varepsilon_j} = 
\frac{\partial g^n}{\partial y_{max}^n} \frac{\partial y_{max}^n}{\partial \varepsilon_j}
\]

And the next derivative can be calculated.
\[
\frac{\partial g^n}{\partial y_{max}^n} =
\begin{cases}
  0 & \langle \sqrt{y_{max}^n} - \sqrt{y_0^n} \rangle_+ > 0 \\
  \frac{1}{2 \sqrt{y_{max}^n} \sqrt{y_c^n}} & else
\end{cases}
\]

Finally, the end of the chain is near. The driving
forces explicitly depend on \(\strain\). Their derivatives are vectorial
are calculated for a given load step \(i\) as
\[
\frac{\partial y_{max}^n}{\partial \varepsilon_j} =
\begin{cases}
  0 & y_i^n > y_{max}^n \\
  \frac{\partial y_{max}^n}{\partial \varepsilon_j} & else
\end{cases}
\]

Then for each driving force, the final derivative  in Voigt notation is
\begin{equation}
  \begin{split}
    \frac{\partial y^1}{\partial \strain} &=
    [ C_{11}^0 \varepsilon_{1} 0 0 0 b^2 C_{55}^0 \varepsilon_{5} b^1 C_{66}^0 \varepsilon_{6} ] \\
    \frac{\partial y^2}{\partial \strain} &=
      [ 0 C_{22}^0 \varepsilon_{2} 0 b^2 C_{44}^0 \varepsilon_{4} 0 b^1 C_{66}^0 \varepsilon_{6} ] 
  \end{split}
\end{equation}

To summarize in broad strokes, the material tangent stiffness tensor
is calculated as follows:
\begin{equation*}
  \begin{split}
    \frac{\partial \sigma_i}{\partial \varepsilon_j} &=
    \frac{\partial \tilde{C}_{ik}}{\partial \varepsilon_j} \varepsilon_k^{*} + \tilde{C}_{ij} \\
     &= -\tilde{C}_{il} \frac{\partial \tilde{S}_{lm}}{\partial \varepsilon_j} \tilde{C}_{mk} \varepsilon_k^* + \tilde{C}_{ij} \\
     &= -\tilde{C}_{il} (\frac{\partial d^1}{\partial \varepsilon_j} H_{lm}^1 + \frac{\partial d^2}{\partial \varepsilon_j} H_{lm}^2)
    \tilde{C}_{mk} \varepsilon_k^* + \tilde{C}_{ij} \\
  \end{split}
\end{equation*}

With the quantities
\begin{equation*}
  \begin{split}
    \frac{\partial d^1}{\partial \varepsilon_j} &=
    \frac{\partial d^1}{\partial g^1} \frac{\partial g^1}{\partial y_{max}^1} \frac{\partial y_{max}^1}{\partial \varepsilon_j} \\
    \frac{\partial d^2}{\partial \varepsilon_j} &=
    \frac{\partial d^2}{\partial g^2} \frac{\partial g^2}{\partial y_{max}^2} \frac{\partial y_{max}^2}{\partial \varepsilon_j} \\
  \end{split}
\end{equation*}

Of which, all of the required derivatives have been calculated.

\item Numerical tangent stiffness matrix
\label{sec:org9cf3c37}

The tangent stiffness matrix can be calculated numerically. This
avoids all of the complicated math shown above, but can add some
computational expense.

If a Taylor series expansion of a stress component is taken at the
current strain,
\[
\sigma_i (\strain + \delta \varepsilon_j) = \sigma_i (\strain) +
\frac{\partial \sigma_i}{\partial \varepsilon_j} (\strain) (\delta \varepsilon_j) + O(\delta \varepsilon_j^2)
\]

Where we have abused notation to show that
\(\strain + \delta \varepsilon_j\) is the current strain perturbed in
the \(j^{th}\) component a magnitude \(\delta \varepsilon\).

This Taylor series can be rearranged to get an approximation of the
material tangent stiffness matrix.
\[
\vect{C}_{tan} = \frac{\partial \sigma_i}{\partial \varepsilon_j} \approx
\frac{\sigma_i (\strain + \delta \varepsilon_j) - \sigma_i (\strain)}{\delta \varepsilon_j}
\]

In practice, for each of the six perturbations of strain a
new vector of stresses are calculated. Those new stresses are
subtracted from the base set of stresses and a column of the tangent
stiffness matrix is obtained.

The numerical tangent stiffness matrix has the same problems with
symmetry as the analytical one. It is also symmetrized to simplify
analysis.

One must also be careful about the value of the strain perturbation. A
smaller value is more accurate, but can lead to numerical troubles.
\item General comments about the tangent stiffness matrix
\label{sec:org36dc872}

In numerical studies with the material, we have found the tangent
stiffness matrix to become non-positive definite. In the material
modeling world, this means the material is softening. In other words,
an increase in strain causes a decrease in stress. This has major
applications on the numerical solution. An arclength solver can be
used to capture general behavior. One can also solve strictly
displacement controlled problems to avoid issues.

Also, both the numerical and analytical tangent stiffness tensors are
not generally symmetric. This has significant impacts on the
computational expense of the method. To remedy this, the tangent
matrix is always symmetrized as follows: 
\[
\vect{C}_{tan}^{sym} = \frac{1}{2} ( \vect{C}_{tan} + \vect{C}_{tan}^T )
\]

This symmetrized tangent is no longer analytical, so we are
effectively restricted to quasi-newton solver methods.
\end{enumerate}

\section{Model Parameters}
\label{sec:org0ac2884}
\begin{table}[h!]
  \label{tab:OpenDM2Mode_ParamDef}
  \centering
  \begin{tabular}{c|c|c}
    Parameter & Number & Effect \\
    \hline
    $h_i^m$ & $6$ & scales damage mode $m$ effect on compliance from fracture mode $i$ \\
    $b_i$ & $2$ & scales contribution of shear strains to driving forces \\
    $y_0^i$ & $2$ & sets a threshold value of the driving force for mode $m$ \\ 
    $y_c^i$ & $2$ & changes the celerity of the damage progression \\
    $d_c^i$ & $2$ & sets the maximum value of the damage parameter \\
    $p^i$ & $2$ &changes the shape of the curve of damage progression \\
  \end{tabular}
\end{table}
\end{document}
