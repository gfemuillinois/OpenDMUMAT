% Created 2024-11-15 Fri 10:32
% Intended LaTeX compiler: pdflatex
\documentclass[11pt]{article}
\usepackage[utf8]{inputenc}
\usepackage[T1]{fontenc}
\usepackage{graphicx}
\usepackage{longtable}
\usepackage{wrapfig}
\usepackage{rotating}
\usepackage[normalem]{ulem}
\usepackage{amsmath}
\usepackage{amssymb}
\usepackage{capt-of}
\usepackage{hyperref}
\usepackage[margin=0.5in]{geometry}
\usepackage{bpmResearchPaper}
\setlength{\parindent}{0pt}
\setlength{\parskip}{5pt}
\author{Bryce Mazurowski}
\date{\today}
\title{OpenDM Damage Model Theory Manual}
\hypersetup{
 pdfauthor={Bryce Mazurowski},
 pdftitle={OpenDM Damage Model Theory Manual},
 pdfkeywords={},
 pdfsubject={},
 pdfcreator={Emacs 29.4 (Org mode 9.6.15)}, 
 pdflang={English}}
\begin{document}

\maketitle

\section{Problem Definition}
\label{sec:orgac83f4b}
\subsection{Solid mechanics: infinitesimal deformation}
\label{sec:orgfbe96c5}
The partial differential equation we wish to solve is the balance of
linear momentum. Stated in strong form neglecting inertial effects and
body forces:
\begin{equation}
  \diverge \stress = 0 \quad \forall \pos \in \Omega
\end{equation}

with boundary conditions
\begin{equation}
  \begin{split}
    \disp &= \bar{\disp} \quad \forall \pos \in \Gamma_d \\
    \stress \dot \vect{n} &= \bar{\traction}
    \quad \forall \pos \in \Gamma_n \\
  \end{split}
\end{equation}

The relationship between stress and strain is
\begin{equation}
  \stress = \bar{\vect{C}} : \strain
\end{equation}

The infinitesimal strain tensor is used
\[
\strain = \frac{1}{2}(\grad \disp + \grad^T \disp)
\]

where \(\disp\) is the displacement solution to the problem.
\subsection{Weak Form}
\label{sec:org5ca83d9}
The method of weighted residuals is used to arrive at the weak form of
the problem:
\begin{equation}
  \int_{\Omega} \grad \weight : \stress (\disp) d\Omega =
  \int_{\Gamma_n} \weight \dot \bar{\traction} d\Gamma
\end{equation}

where \(\weight \in H^1\) is an arbitrary weight function and
\(\disp \in H^1\) is the displacement solution.
\subsection{Material nonlinearity:}
\label{sec:org7e11a1a}
Material nonlinearity requires a special solver. Newton-Raphson is the
classical choice. However, as discussed in Section \ref{sec:org3f97406}, for the OpenDM Damage Models a
Quasi-Newton solver is used. This is because the material tangent stiffness
matrix is not exact. The displacement solution becomes incremental:
\[
\disp_n = \sum_{j=0}^{n} \Delta \disp_j
\]

With updates at each NR solve at each load step:
\[
\disp_{n+1} = \disp_n + \Delta \disp_{n+1}
\]

The weak form can be used to define a residual:
\[
R = 
\int_{\Gamma_n} \weight \dot \bar{\traction} d\Gamma -
\int_{\Omega} \grad \weight : \stress (\disp) d\Omega
\]

Then the residual is linearized via a Taylor Series:
\[
R (\disp_{n+1}) = R (\disp_{n}) +
\frac{\partial R}{\partial \disp} (\disp_{n})
\delta \disp_{n+1} +
O(\delta \disp^2)
\]

Progressing from a converged solution, \(R(\disp_n) = 0\), dropping
higher order terms:
\[
\frac{\partial R}{\partial \disp}(\disp_n) \delta \disp_{n+1} \approx 
R (\disp_{n+1})
\]

The quantity \(\disp_{n+1}\) is found iteratively in the Newton-Raphson
scheme. So we solve a given load step \(n\) for a number of NR
increments \(i\) as:
\[
\frac{\partial R}{\partial \disp} (\disp_n^i) \delta \disp_{n+1}^{i+1} =
R (\disp_{n}^{i})
\]

With our residual, this becomes:
\[
\int_{\Omega} \grad \weight : \frac{\partial \stress}{\partial \disp}
\delta \disp_{n+1}^{i+1} d\Omega =
\int_{\Gamma_n} \weight \dot \bar{\traction}^{n+1} d\Omega -
\int_{\Omega} \grad \weight : \stress (\disp_{n}^i) d\Omega 
\]

Once the system of equations is solved for a given \(i\), the
displacement is updated. The residual is then recomputed, and can be
used as a convergence check. The term \(\frac{\partial
\stress}{\partial \disp}\) is the material tangent stiffness
matrix. This matrix is computed based on the formulation of the OpenDM
material model.

\section{OpenDM Damage Models}
\label{sec:org25fcd9c}
OpenDM damage models are based on the formulation in the papers
(1, 4). These works are
based on the Onera formulation in (2, 3).

To simplify presentation of the models, stresses and strains are given
in Voigt notation, that is:
\begin{equation}
  \label{eq:VoigtStressStrain}
  \begin{split}
    \strain &= [\strainComp{11}, \strainComp{22}, \strainComp{33},
              2 \strainComp{23}, 2 \strainComp{13}, 2 \strainComp{12}]^T =
              [\strainComp{1}, \strainComp{2}, \strainComp{3},
                    \strainComp{4}, \strainComp{5}, \strainComp{6}]^T \\
    \stress &= [\stressComp{11}, \stressComp{22}, \stressComp{33},
              \stressComp{23}, \stressComp{13}, \stressComp{12}]^T =
              [\stressComp{1}, \stressComp{2}, \stressComp{3},
              \stressComp{4}, \stressComp{5}, \stressComp{6}]^T \\
  \end{split}
\end{equation}

All quantities are also given in material coordinates \(\hat{\pos}\).

The Helmholtz Free Energy is defined as
\begin{equation}
  \label{eq:OpenDM_Helmholtz}
  \psi = \frac{1}{2 \rho}
  (\strain^* \dot \bar{\vect{C}} \strain^*),
\end{equation}

where \(\rho\) is density, \(\strain^*\) is the mechanical strain,
and \(\bar{\vect{C}}\) effective material stiffness.

For now, thermal strain \(\strain^{th}\) is taken to always be
zero, this is an area ripe for future work. Material stress behavior
is then the derivative of the Helmholtz free energy with respect to
the strain,
\begin{equation}
  \label{eq:OpenDM_Stress}
  \stress = \bar{\vect{C}} \strain^*.
\end{equation}

The material stiffness tensor is defined as
\begin{equation}
  \label{eq:OpenDM_Stiffness}
  \bar{\vect{C}} = (\bar{\vect{S}})^{-1},
\end{equation}

where \(\bar{\vect{S}}\) is the effective material compliance
tensor. The definition of \(\bar{\vect{S}}\) is where the damage modes
begin to appear 
\begin{equation}
  \label{eq:OpenDM_EffCompliance}
  \bar{\vect{S}} = \vect{S}^0 + \sum_{n=1}^2 d^n \vect{H}^n,
\end{equation}

where \(\vect{S}^0\) is the undamaged material stiffness tensor and
\(d^n\) is the damage variable for mode \(n=1,2,4,5\) with the final two
removed in the 2-Mode model. \(\vect{H}^n, n=1,2,4,5\) are constants
that modify material compliance these tensors are defined for their
respective damage model below. The 2-Mode model makes use of
\(\vect{H}^1\) and \(\vect{H}^2\) only.

Energy dissipation as a result of matrix microcracking is captured by
independent dissipation potentials for each damage mode. Each mode has
the same form,
\begin{equation}
  \label{eq:OpenDM_DamagePotential}
  F^n = f(y^n) - d^n.
\end{equation}

The driving forces \(y^n\) vary for each mode and model and are defined
below.The definition of \(f(y^n)\) is carried out over two steps, first
\begin{equation}
  \label{eq:OpenDM_gVals}
  g^n =
  \frac{\langle \sqrt{y_{max}^n} - \sqrt{y_0^n}\rangle_+}{
  \sqrt{y_c^n}},
\end{equation}

where \(y_{max}^n\) adds a history effect to the model, which takes the
largest \(y^n\) value over time for each mode calculation at a given
material point and \(\langle \dot \rangle_+\) is the positive Macaulay
bracket. Model parameter \(y_0^n\) sets a threshold
value of the driving force to start damage. Another parameter \(y_c^n\)
changes the celerity of the damage progression. Then, the definition
of \(f(y^n)\) is
\begin{equation}
  \label{eq:OpenDM_DamageEvol}
  f(y^n) = d_c^n [1 - \exp(-(g^n)^{(p^n)})].
\end{equation}

An additional parameter \(d_c^n\) sets the maximum value that a given
damage mode can reach and a final parameter \(p^n\) changes the shape of
the curve of damage progression.

\subsection{2-Mode Model}
\label{sec:orga8f17f0}
Matrix microcracking perpendicular to each fiber direction is captured
by the 2-Mode model. Matrix microcrack growth for each mode is shown
in Figure \ref{fig:OpenDM2_DamageModes}.
\begin{figure}[h!]
  \centering
  \def\svgwidth{0.5\linewidth}
  \input{OpenDM2_DamageModes.pdf_tex}
  \caption{Damage modes in material coordinates $\hat{\pos}$ for the OpenDM 2-Mode
    damage model. Representative matrix microcracks are shown for each
    damage mode.}
  \label{fig:OpenDM2_DamageModes}
\end{figure}

The definition of \(\vect{H}^1\) and \(\vect{H}^2\) is
\begin{equation}
  \label{eq:OpenDM2_HTens}
  \vect{H}^1 =
  \begin{bmatrix}
    \eta^1 h_1^1 S_{11}^0 & 0 & 0 & 0 & 0 & 0 \\
    0 & 0 & 0 & 0 & 0 & 0 \\
    0 & 0 & 0 & 0 & 0 & 0 \\
    0 & 0 & 0 & 0 & 0 & 0 \\
    0 & 0 & 0 & 0 & h_3^1 S_{55}^0 & 0 \\
    0 & 0 & 0 & 0 & 0 & h_2^1 S_{66}^0 \\
  \end{bmatrix}, \
  %
  \vect{H}^2 =
  \begin{bmatrix}
    0 & 0 & 0 & 0 & 0 & 0 \\
    0 & \eta^2 h_1^2 S_{22} & 0 & 0 & 0 & 0 \\
    0 & 0 & 0 & 0 & 0 & 0 \\
    0 & 0 & 0 & h_3^2 S_{44} & 0 & 0 \\
    0 & 0 & 0 & 0 & 0 & 0 \\
    0 & 0 & 0 & 0 & 0 & h_2^2 S_{66} \\
  \end{bmatrix},
\end{equation}

where \(h_1^n\), \(h_2^n\), and \(h_3^n\) are model parameters that scale
the effect of each fracture mode on material compliance for damage
mode \(n\). The nonzero parts of \(\vect{H}^n\) represent effect of Mode
I, II, III microcrack growth associated with each mode. Microcrack
closure is captured by the function \(\eta^n\) for each mode \(n\). In
(3), a progressive deactivation rule is used to
represent microcracks closing at different times. Here a simple on-off
function based on the current strain is used
\begin{equation}
  \label{eq:DamageDeactivation}
  \eta^n = 
  \begin{cases}
    1 & \strain^n \geq 0 \\
    0 & else
  \end{cases},
\end{equation}

where \(\strain^n\) is the normal strain associated with Mode I fracture
for each damage mode. In Equation (\ref{eq:OpenDM2_HTens}), there is
only damage deactivation for Mode I. This is consistent with closed
cracks, as shear stresses (Mode II and III) can still affect the
stiffness under compressive normal loads.

The driving forces for modes \(1\) and \(2\) are then
\begin{equation}
  \label{eq:OpenDM2_DrivingForces}
  \begin{split}
    y^1 &= \frac{1}{2} (
    \varepsilon_{1}^+ C^0_{11} \varepsilon_{1}^+ + 
    b_2 \varepsilon_{5} C^0_{55} \varepsilon_{5} + 
    b_1 \varepsilon_{6} C^0_{66} \varepsilon_{6} ), \\ 
    y^2 &= \frac{1}{2} (
    \varepsilon_{2}^+ C^0_{22} \varepsilon_{2}^+ + 
    b_2 \varepsilon_{4} C^0_{44} \varepsilon_{4} + 
    b_1 \varepsilon_{6} C^0_{66} \varepsilon_{6} ), \\ 
  \end{split}
\end{equation}

where \(\vect{C}^0\) is the undamaged stiffness tensor and \(b_i, i=1,2\), are
mode-specific shear coupling parameters that increase the effect shear
strains have on damage growth.

The quantities \(\strainComp{1}^+\) and \(\strainComp{2}^+\) indicate the
positive part of the normal strain. A Macaulay bracket is used to
calculate this for each strain component. This assures that
compressive normal force does not contribute to the driving force for
either damage mode, which is consistent with microcrack closure.

\subsection{4-Mode Model}
\label{sec:org644d031}
In the 4-mode model, the first two damage modes represent the same
type of microcracking as in the previous model. The additional two
capture matrix microcracks that are oriented \(45^\circ\) in either
direction from the fiber direction. The four damage modes are shown in
Figure \ref{fig:OpenDM4_DamageModes}.
\begin{figure}[h!]
  \centering
  \def\svgwidth{0.8\linewidth}
  \input{OpenDM4_DamageModes.pdf_tex}
  \caption{Damage modes in material coordinates $\hat{\pos}$ for the OpenDM 4-Mode
    damage model. Representative matrix microcracks are shown for each
    damage mode.}
  \label{fig:OpenDM4_DamageModes}
\end{figure}

The definition of \(\vect{H}^1\) and \(\vect{H}^2\) from
Equation (\ref{eq:OpenDM2_HTens}) are used again. For \(d^4\) and \(d^5\),
a new definition is given. Since this damage is not aligned with the
material coordinates, the compliance tensor is not orthotropic, but
anisotropic. A rotated compliance tensor \(\bar{\vect{S}}\) is used,
transformed either by a \(45^\circ\) rotation about the \(\hat{\pos}_3\)
axis for \(d^4\) or a \(-45^\circ\) rotation about the \(\hat{\pos}_3\) axis
for \(d^5\). The tensors \(\hat{\vect{H}}^n, n = 4,5\) is defined in the
respective coordinate system as
\begin{equation}
  \label{eq:OpenDM4_HTens}
  \bar{\vect{H}}^n =
  \begin{bmatrix}
    \eta^n h_1^n \bar{S}_{11}^0 & 0 & 0 & 0 & h_4^n \bar{S}_{15}^0 & h_4^n \bar{S}_{16}^0 \\
    0 & 0 & 0 & 0 & 0 & 0 \\
    0 & 0 & 0 & 0 & 0 & 0 \\
    0 & 0 & 0 & 0 & 0 & 0 \\
    h_4^n \bar{S}_{15}^0 & 0 & 0 & 0 & h_3^n \bar{S}_{55}^0 & 0 \\
    h_4^n \bar{S}_{16}^0 & 0 & 0 & 0 & 0 & h_2^n \bar{S}_{66}^0 \\
  \end{bmatrix},
\end{equation}

where \(\bar{\vect{S}}^0\) is the undamaged compliance tensor in damage
coordinates. \(\bar{\vect{H}}^n\) needs to be transformed back to material
coordinates before use. With a proper transformation matrix
\(\vect{T}^n\) for the corresponding damage mode coordinate
system, \(\vect{H}^n, n = 4,5\) is defined as
\begin{equation}
  \label{eq:OpenDM4_HTrans}
  \vect{H}^n = \vect{T}^{\vect{S},n} \bar{\vect{H}}^n (\vect{T}^{\vect{S},n})^T
\end{equation}

where \(h_1^n\), \(h_2^n\), \(h_3^n\), and \(h_4^n\) are model parameters that scale
the effect of each fracture mode on material compliance for damage
mode \(n\) and \(\vect{T}^{\vect{S},n}\) is the transformation from the
compliance tensor that transforms from the damage coordiantes to
material coordinates. The additional parameter \(h_4^n\), which captures
the now nonzero contribution to Mode II and III fracture from the
rotated compliance tensor. The \(\vect{H}^n\) matrices represent the
same effects as in the 2-Mode model, that is increased compliance from
Mode I, II, and III fracture. Equation (\ref{eq:DamageDeactivation})
is again used to capture microcrack closure for each mode. In the
\(d^4\) and \(d^5\) case, the normal strain in the rotated coordinates are
used to turn the damage on or off. 

The driving forces for this model are considerably more
complicated. They are defined in two steps, the first is a strain
energy type quantity 
\begin{equation}
  \label{eq:OpenDM4_DrivingForceSE}
  \begin{split}
    z_1 &= \frac{1}{2} ( \strainComp{1}^{d^{1+}} \vect{C}_{11}^0 \strainComp{1}^{d^{1+}} + 
          b_2 \strainComp{5}^{d^{1+}} \vect{C}_{55}^0 \strainComp{5}^{d^{1+}} + 
          b_1 \strainComp{6}^{d^{1+}} \vect{C}_{66}^0 \strainComp{6}^{d^{1+}} ), \\  
    z_2 &= \frac{1}{2} ( \strainComp{2}^{d^{2+}} \vect{C}_{11}^0 \strainComp{1}^{d^{2+}} + 
          b_2 \strainComp{4}^{d^{2+}} \vect{C}_{44}^0 \strainComp{4}^{d^{2+}} + 
          b_1 \strainComp{6}^{d^{2+}} \vect{C}_{66}^0 \strainComp{6}^{d^{2+}} ), \\
    z_6 &= \frac{1}{4} ( \strainComp{1}^{d^{1+}} \vect{C}_{11}^0 \strainComp{6}^{d^{1+}} + 
          \strainComp{2}^{d^{2+}} \vect{C}_{22}^0 \strainComp{6}^{d^{2+}} + 
          b_3 \vect{C}_{66}^0 (\strainComp{6}^{d^{1+}} \strainComp{1}^{d^{1+}} +
          \strainComp{6}^{d^{2+}} \strainComp{2}^{d^{2+}} ),
  \end{split}
\end{equation}

where \(\strain^{d^{1+}}\) and \(\strain^{d^{2+}}\) are the positive parts
of the strain tensor for the \(d^1\) and \(d^2\) modes, and \(b_3\) is an
additional parameter to control the effect of shear coupling in
driving damage. Each mode's strain decomposition is
\begin{equation}
  \label{eq:OpenDM4_StrainSplit}
  \begin{split}
    \strain^{d^1} &= [\strainComp{1}, 0, 0, \strainComp{5}, \strainComp{6}]^T \\
    \strain^{d^2} &= [0, \strainComp{2}, 0, \strainComp{4}, 0, \strainComp{6}]^T. \\
  \end{split}
\end{equation}

A Macaulay bracket on the eigenvalues of the strain is used to compute
the positive part of the strain, as discussed in
(5). The calculation is as follows
\begin{equation}
  \label{eq:OpenDM4_PosPartStrain}
  \strainComp{i}^+ = P_{ij} \langle \hat{\strainComp{j}} \rangle_+,
\end{equation}

where \(\hat{\strain}\) is the vector of eigenvalues of the strain
tensor and \(\vect{P}\) is the projection tensor that transforms strain
from the spectral basis to the material basis.

The driving forces used in Equation (\ref{eq:OpenDM_DamagePotential})
are then
\begin{equation}
  \label{eq:OpenDM4_DrivingForces}
  \begin{split}
    y^1 &= z_1 - |z_6| \\ 
    y^2 &= z_2 - |z_6| \\ 
    y^4 &= \langle z_6 \rangle_+ \\ 
    y^5 &= \langle z_6 \rangle_- \\ 
  \end{split}
\end{equation}

\subsection{Tangent Stiffness Tensor}
\label{sec:org6cca6b1}
For both the 2-Mode and 4-Mode models, the numerical tangent stiffness
tensor is used. The analytical tangent is certainly tractable for the
2-Mode model, but the 4-Mode model is quite complicated. In the 4-Mode
model, behavior depends on the eigenvalues, which requires a
transformation. The transformation matrix then depends on the
eigenvectors. To simplify this process, the numerical tangent is used.

The numerical tangent is based on a Taylor series approximation of the
stress field
\begin{equation}
  \label{eq:OpenDM_TaylorStress}
  \stressComp{i} (\strain + \delta \strainComp{j}) =
  \stressComp{i} (\strain) +
  \frac{\partial \stressComp{i}}{\partial \strainComp{j}} (\strain)
  (\delta \strainComp{j}) + O(\delta \strainComp{j}^2),
\end{equation}

where there is a bit of notation abuse in
\(\strain + \delta \strainComp{j}\), which is meant to represent the
current strain perturbed in the \$j\$-th component by a magnitude of
\(\delta \strainComp{}\).

Equation\textasciitilde{}(\ref{eq:OpenDM_TaylorStress}) is then rearranged to get the
tangent stiffness tensor \(\vect{C}^{tan}\)
\begin{equation}
  \label{eq:OpenDM_NumTang}
  \vect{C}_{ij}^{tan} = \frac{\partial \stressComp{i}}{\partial \strainComp{j}} =
  \frac{\stressComp{i} (\strain + \delta \strainComp{j}) - \stressComp{i} (\strain)}{
    \delta \strainComp{}}.
\end{equation}

In words, for each perturbation of strain the difference in stress
response is calculated, and then normalized by the magnitude of
the perturbation. Each strain perturbation corresponds to a column of
the tangent stiffness tensor.

Both the 2-Mode and 4-Mode tangent tensor can be nonsymmetric. This is
quite cumbersome to deal with numerically, in terms of solver time and
memory, so the tangent is symmetrized
\begin{equation}
  \label{eq:OpenDM_SymTang}
  \hat{\vect{C}}^{tan} = \frac{1}{2} (\vect{C}^{tan} + (\vect{C}^{tan})^T).
\end{equation}

This does mean that the tangent stiffness tensor is not exact, and
thus a Newton-Raphson solver becomes a quasi-newton solver. Also, this
generally has some effect on the convergence of the system, but the
symmetrized tensor likely makes up for this with a quicker and less
memory-dense solver. 

\subsubsection{General comments about the tangent stiffness matrix}
\label{sec:org3f97406}
Both the 2-Mode and 4-Mode tangent tensor can be nonsymmetric. This is
quite cumbersome to deal with numerically, in terms of solver time and
memory, so the tangent is symmetrized
\begin{equation}
  \label{eq:OpenDM_SymTang}
  \hat{\vect{C}}^{tan} = \frac{1}{2} (\vect{C}^{tan} + (\vect{C}^{tan})^T).
\end{equation}

This does mean that the tangent stiffness tensor is not exact, and
thus a Newton-Raphson solver becomes a quasi-newton solver. Also, this
generally has some effect on the convergence of the system, but the
symmetrized tensor likely makes up for this with a quicker and less
memory-dense solver. 

In the numerical explorations of the 2-Mode and 4-Mode damage models,
both have exhibited softening. This occurs as a result of combined
damage modes and manifests as negative eigenvalues in the tangent
stiffness tensor. This has been more of a problem in the 4-Mode model,
as is evidenced in Section\textasciitilde{}\ref{sec:ModelVal}. Several attempts to
remedy this have been unsuccessful, including using the nonsymmetric
tangent tensor, using an arclength solver, smaller load steps, and
attempts to change convergence tolerances in Abaqus. This is not an
issue if only a single damage mode is active at a time. It is an issue
in the validation tests where the response is more complicated. To
remedy this, problems should be constructed to be
displacement-controlled.


\section{References}
\label{sec:org3207239}
\noindent
[1] Craig Przybyla and Antoine D\'{e}barre and Jean-Fran\c{c}ois Maire and Emmanuel Baranger and Fr\'{e}d\'{e}ric Laurin, \emph{Scaler nonlinear continuum damage models for ceramic matrix composites with and without in plane ply anisotropy at room temperature}, Journal of European Ceramic Society, 2025.

\noindent
[2] J. L. Chaboche and J. F. Maire, \emph{A new micromechanics based {CDM} model and its application to {CMC}'s}, Aerospace Science and Technology, 2002.

\noindent
[3] Lionel Marcin and Jean-Francois Maire and Nicolas Carr\`{e}re and Eric Martin, \emph{Development of a Macroscopic Damage Model for Woven Ceramic Matrix Composites}, International Journal of Damage Mechanics, 2011.

\noindent
[4] Mazurowski, Bryce and Przybyla, Craig and C. Armando Duarte, \emph{Multiscale analysis of open holes and fasteners in {CMC} structures with the Generalized Finite Element Method}, Composite Structures, Submitted.

\noindent
[5] Murakami, Sumio, \emph{Continuum damage mechanics: a continuum mechanics approach to the analysis of damage and fracture}, Springer Science \& Business Media, 2012.
\end{document}
